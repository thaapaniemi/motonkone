 %----------------------------------------------------------------------------------------
%	Metropolia Thesis LaTeX Template
%----------------------------------------------------------------------------------------
% License:
% This work is licensed under the Creative Commons Attribution 4.0 International License. To view a copy of this license, visit http://creativecommons.org/licenses/by/4.0/.
%
% Authors:
% Panu Leppäniemi, Patrik Luoto and Patrick Ausderau
%
% Credits:
% Panu Leppäniemi: abstract, def, cleaning,...
% Patrik Luoto: title page, abstract in Finnish, abbreviation, math,...
% Patrick Ausderau: initial version, style, table of content, bibliography, figure, appendix, table, source code listing...
%
% Please:
% If you find mistakes, improve this template and alike, please contribute by sharing your improvements and/or send us your feedback there: https://github.com/panunu/metropolia-thesis-latex
% And of course, if you improve it, add yourself as an author.
%
% Compiler:
% Use XeLaTeX as a compiler.

%----------------------------------------------------------------------------------------
%	THESIS
%----------------------------------------------------------------------------------------

\def\thesislang{finnish} %change this depending on your language
\author{Tomi Haapaniemi}
\def\thesis{Insinöörityö/Thesis}
\def\alaotsikko{Sunit Nero / Valmet 901 II}
%\def\alaotsikko{Alaotsikko/Subtitle}

%Finnish section
\def\otsikko{Ajoneuvotietokoneen päivitysmahdollisuudet hakkuukoneessa}
\def\tiivistelmaotsikko{ \otsikko : \alaotsikko}
\def\tutkinto{Insinööri (AMK)}
\def\kohjelma{Tietotekniikan koulutusohjelma}
\def\suuntautumis{Ohjelmistotekniikka/Sulautettu tietotekniikka}
\def\ohjaajat{
Keijo Länsikunnas, Lehtori%\newline
}
\def\avainsanat{Hakkuukone, ajoneuvotietokone}
\def\pvm{\ddmmyyyydate\today}

%English section, for abstract
\title{Possibilities to upgrade Sunit Nero -embedded computer in Valmet 901 II -Harvester}
\def\metropoliadegree {Bachelor of Engineering}
\def\metropoliadegreeprogramme {Degree Programme in Information Technology}
\def\metropoliaspecialisation {Software Engineering/Embedded Engineering}
\def\metropoliainstructors {
Keijo Länsikunnas, Lecturer %(for example: Project Manager)\newline
%First name Last name, Title (for example: Principal Lecturer)
}
\def\metropoliakeywords {harvester, in-vehicle computer}
\date{\today}

%----------------------------------------------------------------------------------------
%	GLOBAL STYLES
%----------------------------------------------------------------------------------------

\documentclass[11pt,a4paper,oneside,article]{memoir}
\usepackage[\thesislang]{babel} 
\usepackage{float}
\usepackage{iflang}
\usepackage{amsmath}
\usepackage{amsfonts}
\usepackage{amssymb}
\usepackage{fontspec}
\usepackage{tocloft}
\usepackage{titlesec}
\usepackage[hyphens]{url}
\usepackage{mathtools}
\usepackage{wallpaper}
\usepackage{datetime}
\usepackage[bookmarksdepth=subsection]{hyperref} % for automagic pdf links for toc, refs, etc.
\usepackage[amssymb]{SIunits}
\usepackage[version=3]{mhchem}
\usepackage{pgfplots} %simple plots etc
\usepackage{pgfplotstable}
\usepackage{tikz} % mindmaps, flowcharts, piecharts, examples at http://www.texample.net/tikz/examples/
\usetikzlibrary{shapes.geometric, arrows}


\renewcommand{\dateseparator}{.}
%condition for adding or not space in TOC
\usepackage{etoolbox}
%for compact list
\usepackage{enumitem}
%for block comment
\usepackage{verbatim}
%for "easier" references
\usepackage{varioref}
%forcing single line spacing in bibliography
\DisemulatePackage{setspace}
\usepackage{setspace}
%including figure (image)
\usepackage{graphicx}
%change the numbering for figure
\usepackage{chngcntr}
%strike trough
\usepackage{ulem}
%euro symbol
\usepackage{eurosym}
%try to count
\usepackage{totcount}
%insert source code
\usepackage{listings}
\usepackage{caption}
\usepackage{color}
%force the width of a table instead of column
\usepackage{tabularx}
\usepackage{booktabs} %why not booktabs? :3

\newcommand\tn[1]{\textnormal{#1}} %use \tn instead of \textnormal
\newcommand\reaction[1]{\begin{equation}\ce{#1}\end{equation}} %\reaction{} for chemical reactions

%NORMAL TEXT
%all text, title, etc. in the same font: Arial
%replace with arial.ttf if you have the fontfile
%NOTE: fontname is case-sensitive
\setmainfont
[BoldFont=LiberationSans-Bold.ttf,
ItalicFont=LiberationSans-Italic.ttf,
BoldItalicFont=LiberationSans-BoldItalic.ttf]
{LiberationSans-Regular.ttf}
%line space
\linespread{1.5}
%\doublespacing
%margin
\usepackage[top=2.5cm, bottom=3cm, left=4cm, right=2cm, nofoot]{geometry}
\setlength{\parindent}{0pt} %first line of paragraph not indented
\setlength{\parskip}{16.5pt} %one empty line to separate paragraph
%list with small line space separation
\tightlists

%IMAGE - FIGURE
%the figures should be placed in the "illustration" folder
\graphicspath{{illustration/}}
%figure number without chapter (1.1, 1.2, 2.1) to (1, 2, 3)
\counterwithout{figure}{chapter}
%border around images
\setlength\fboxsep{0pt}
\setlength\fboxrule{0.5pt}
%caption font size
\captionnamefont{\small}
\captiontitlefont{\small}
%space after figure caption (and other float elements)
\setlength{\belowcaptionskip}{-7pt}

%TABLE
\counterwithout{table}{chapter}

%SOURCE CODE
\definecolor{darkgray}{rgb}{.4,.4,.4}
\definecolor{purple}{rgb}{0.65, 0.12, 0.82}
\lstset{
extendedchars=true,
captionpos=b,
caption=\footnotesize,
basicstyle=\singlespacing\ttfamily,%\small\fontfamily{"Courier"}\selectfont,
keywordstyle=\color{blue}\bfseries,
commentstyle=\color{purple}\itshape,
identifierstyle=\color{black},
stringstyle=\color{red},
showstringspaces=false,
showspaces=false,
numbers=left,
numberstyle=\footnotesize,
numbersep=9pt,
breaklines=true,
tabsize=2,
showtabs=false,
xleftmargin=1cm
}
\IfLanguageName {finnish} {\renewcommand{\lstlistingname}{Listaus}} {} % what is a good translation for this?
%\counterwithout{lstlisting}{chapter}
%moved after begin document, otherwise does not compile

%TOC
%change toc title
\IfLanguageName {finnish} {\addto{\captionsfinnish}{\renewcommand*{\contentsname}{Sisällys}}} {}
%remove dots
\renewcommand*{\cftdotsep}{\cftnodots}
%chapter title and page number not in bold
\renewcommand{\cftchapterfont}{}
\renewcommand{\cftchapterpagefont}{}
%sub section in toc
\setcounter{tocdepth}{2}
%subsection numbered
\setcounter{secnumdepth}{2}
\renewcommand{\tocheadstart}{\vspace*{-15pt}}
\renewcommand{\printtoctitle}[1]{\fontsize{13pt}{13pt}\bfseries #1}
\renewcommand{\aftertoctitle}{\vspace*{-22pt}\afterchaptertitle}
%spacing afer a chapter in toc
\preto\section{%
  \ifnum\value{section}=0\addtocontents{toc}{\vskip11pt}\fi
}
%spacing afer a section in toc
\renewcommand{\cftsectionaftersnumb}{\vspace*{-3pt}}
%spacing afer a subsection in toc
\renewcommand{\cftsubsectionaftersnumb}{\vspace*{-1pt}}
%appendix in toc with "Appendix " + num
\IfLanguageName {finnish} {
  \renewcommand*{\cftappendixname}{Liite\space}
  \renewcommand{\appendixtocname}{Liitteet}
}{\renewcommand*{\cftappendixname}{Appendix\space}}
%appendix header
\IfLanguageName {finnish} {\def\appname{Liite\space}}{\def\appname{Appendix\space}}

%TITLES
%chapter title
\titleformat{\chapter}
{\fontsize{13pt}{13pt}\bfseries\linespread{1}}
{\thechapter}{.5cm}{}
\titlespacing*{\chapter}{0pt}{.32cm}{9pt}
\titleformat{\section}
{\fontsize{12pt}{12pt}\linespread{1}}
{\thesection}{.5cm}{}
\titlespacing*{\section}{0pt}{14pt}{6pt}
\titleformat{\subsection}
{\fontsize{12pt}{12pt}\linespread{1}}
{\thesubsection}{.5cm}{}
\titlespacing*{\subsection}{0pt}{14pt}{6pt}


%QUOTE
\renewenvironment{quote}
  {\list{}{\rightmargin=0pt\leftmargin=1cm\topsep=-10pt}%
  \item\relax\fontsize{10pt}{10pt}\singlespacing}
  {\endlist}

%BIBLIOGRAPHY
%bibliography title to be "references"
\IfLanguageName {finnish} {\addto{\captionsfinnish}{\renewcommand*{\bibname}{Lähteet}}} {\renewcommand\bibname{References}}
\makeatletter %reference list option change
\renewcommand\@biblabel[1]{#1\hspace{1cm}} %from [1] to 1 with 1cm gap
\makeatother %
\setlength{\bibitemsep}{11pt}

%count the appendices (since the chapter counter is reset after \appendix).
%! require to complie 2 times
\regtotcounter{chapter}

%TITLE PAGE
\makeatletter
\renewcommand{\maketitle}{
\thispagestyle{empty}
\ThisCenterWallPaper{1}{viiva}
%
\vspace*{9.5cm}
\tn{\LARGE\@author\\[22pt]\Huge\IfLanguageName {finnish}{\otsikko}{\@title}\\[22pt]\LARGE\alaotsikko\\[1.75cm]}

\parbox{.7\linewidth}{
\IfLanguageName {finnish}{
  Metropolia Ammattikorkeakoulu\\
  \tutkinto \\
  \kohjelma \\
  \thesis\\
  \pvm
} {
  Helsinki Metropolia University of Applied Sciences\\
  \metropoliadegree \\
  \metropoliadegreeprogramme \\
  \thesis\\
  \ddmmyyyydate\today %to be checked date format? 
}
}
\ThisLRCornerWallPaper{1}{metropolia}
%
\clearpage
}
\makeatother

\makepagestyle{tiivis}
\makeevenhead{tiivis}{}{}{Tiivistelmä}
\makeoddhead{tiivis}{}{}{Tiivistelmä}

\makepagestyle{abstract}
\makeevenhead{abstract}{}{}{Abstract}
\makeoddhead{abstract}{}{}{Abstract}

\begin{document}
\counterwithout{lstlisting}{chapter}

%page number always on the top right, clear the "chapter/section" head
\pagestyle{myheadings}
\markright{}
%clear chapter "title" foot page
\makeevenfoot{plain}{}{}{}
\makeoddfoot{plain}{}{}{}

%----------------------------------------------------------------------------------------
%	TITLE PAGE
%----------------------------------------------------------------------------------------

\maketitle
\newpage
\LRCornerWallPaper{1}{footer}

%----------------------------------------------------------------------------------------
%    Tiivistelmä
%----------------------------------------------------------------------------------------

\thispagestyle{tiivis}
\begin{tabular}{ | p{4,7cm} | p{10,3cm} |}
  \hline
  Tekijä(t) \newline
  Otsikko \newline\newline

  Sivumäärä \newline
  Aika 
  & 
  \makeatletter
  \@author \newline 
  \tiivistelmaotsikko \newline\newline   
  \makeatother
  \pageref*{LastPage} sivua + \total{chapter} liitettä \newline %! if no appendices, risk to count total of chapter :D
  \pvm		
  \\ \hline
  Tutkinto & \tutkinto
  \\ \hline
  Koulutusohjelma & \kohjelma
  \\ \hline
  Suuntautumisvaihtoehto & \suuntautumis
  \\ \hline
  Ohjaaja(t) & \ohjaajat
  \\ \hline
  \multicolumn{2}{|p{15cm}|}{\begin{singlespacing}\vspace{-22pt}
  %Tämä on tiivistelmän ensimmäinen kappale. Tiivistelmän kappaleet loppuvat komentoon newline, jotta saadaan yksi tyhjä rivi aikaiseksi. \newline
  Tämän insinöörityön tarkoituksena on selvittää, onko mahdollista korvata 17 vuotta vanhan Valmet 901 II- hakkuukoneen ajoneuvotietokone uudemmalla siten, ettei hakkuukoneen järjestelmiin tarvitse koskea, vaan korvaavassa ajoneuvotietokoneessa toimisivat alkuperäisessä ajoneuvotietokoneessa käytetyt ohjelmistot.\newline

  Työ tehtiin yksityiselle yrittäjälle. Työn tarkoituksena on tuottaa lisää tietoa käytössä olevan hakkuukoneen käyttöiän ennusteesta koneen tietotekniikan osalta. Tavoitteena oli saada hakkuukoneeseen toimintakuntoinen, fyysisiltä ulkomitoiltaan yhteensopiva ajoneuvotietokone alkuperäisen tilalle. Koska alkuperäinen järjestelmä osoittautuikin vielä korjauskelpoiseksi ja korvaavan järjestelmän tuottamisen arvioitiin vievän liikaa aikaa, päädyttiin insinöörityö rajaamaan selvitykseksi järkevän budjetin ja aikataulun rajoissa olevasta päivitysmahdollisuudesta.\newline

  Haasteita työlle asetti hakkuukoneen sijainti, joka oli insinöörityön aikana noin 30 km metsään Oriveden keskustasta. Orivedelle on matkaa kotoani Kirkkonummelta noin 200 kilometriä. Insinöörityön teon aikana käytiin alkuperäisen ajoneuvotietokoneen huoltojen yhteydessä hakkuukoneen luona yhteensä kolme kertaa. Hakkuukoneella toteutettujen huoltojen lisäksi laitetta ja sen osia huollettiin etätyönä yrittäjän lähetettyä ne postitse, jolloin matka Orivedelle ei ollut välttämätön.\newline

  Työssä selviää, että hakkuukoneen järjestelmien kommunikointi ajoneuvotietokoneen kanssa tapahtuu standardin RS-232 -sarjaväylän kautta ja käytettävät ohjelmat on mahdollista saada toimimaan uudemmissa käyttöjärjestelmissä. Tämä mahdollistaa uudempien ajoneuvotietokoneiden tai fyysiseltä kestävyydeltään vastaavien tietokoneiden käytön hakkuukoneessa ilman muutoksia hakkuukoneen järjestelmiin.\newline


  
 % Tämä on tiivistlemän toinen kappale.
  \end{singlespacing}} \\[14cm] \hline
  Avainsanat & \avainsanat
  \\ \hline
\end{tabular}
\clearpage

%----------------------------------------------------------------------------------------
%	ABSTRACT
%----------------------------------------------------------------------------------------

\pagestyle{abstract}
\begin{tabular}{ | p{4,7cm} | p{10,3cm} |}
  \hline
  Author(s) \newline
  Title \newline\newline 
  Number of Pages \newline
  Date
  & 
  \makeatletter
  \@author \newline
  \@title \newline
  \pageref*{LastPage} pages + \total{chapter} appendices \newline %! if no appendices, risk to count total of chapter :D
  \IfLanguageName {finnish} {\foreignlanguage{english}{\longdate\@date}} {\@date}
  \makeatother
  \\ \hline
  Degree & \metropoliadegree
  \\ \hline
  Degree Programme & \metropoliadegreeprogramme
  \\ \hline
  Specialisation option & \metropoliaspecialisation
  \\ \hline
  Instructor(s) & \metropoliainstructors
  \\ \hline
  \multicolumn{2}{|p{15cm}|}{\begin{singlespacing}\vspace{-22pt}
  
The purpose of this thesis is to find out whether it is possible to replace the original computer in 17 y. old harvester with a newer computer so that there is no need to do any changes to the harvester's electronical systems. Therefore original communication protocols and controlling software must work in the new computer. User Experience of system with the new computer should be similar with the original computer.\newline

This thesis was done for a private entrepreneur and the results are intended to provide additional information about the life expectation of harvester from the viewpoint of computer systems. The original goal of the thesis was to produce a drop in -replacement computer for the original systems, but because the original system proved to be repairable and the time estimation for producing a durable replacement was too high it was decided to limit the thesis to be only a report about the possibility to upgrade the original system with reasonable budget and time.\newline

The location of the harvester, which was approx. 30 km to forest from the Orivesi city center set some challenges for the thesis. Distance to Orivesi is approx. 200 kilometres from my home at Kirkkonummi. Harvester was visited total of three times. All visits took place after maintenance of the harvester's original computer. In addition the entrepreneur sent some parts for maintenance by mail so that a visit to Orivesi wasn't necessary.\newline

Result of the thesis is that all required communication with harvester is through standard RS-232 serial bus. All required software was proven to work in newer operating systems which allows the possible new system to be selected from a broader range of availabe rugged computer models without any modifications to harvester's electronical systems.\newline


  \end{singlespacing}} \\[14cm] \hline
  Keywords & \metropoliakeywords
  \\ \hline
\end{tabular}
\clearpage

%----------------------------------------------------------------------------------------
%	Acknowledgement ?
%----------------------------------------------------------------------------------------
%\chapter*{Acknowledgement}
%Thanks to my cat
%\clearpage

%----------------------------------------------------------------------------------------
%	TABLE OF CONTENTS
%----------------------------------------------------------------------------------------

\makeevenhead{plain}{}{}{}
\makeoddhead{plain}{}{}{}
\pagestyle{empty} %remove page number in toc (if longer than 2 pages)
\tableofcontents*
\pagestyle{empty} %remove page number in toc (if longer than 1 pages)
\clearpage
\pagestyle{plain}

%list of figure, tables comes here...


%----------------------------------------------------------------------------------------
%    Lyhenteet / Abbreviation
%----------------------------------------------------------------------------------------

\pagestyle{empty}
\setlength{\parskip}{1cm}
\IfLanguageName {finnish} {
  \chapter*{Lyhenteet}
  \cftaddtitleline{toc}{chapter}{Lyhenteet}{}
} {
  \chapter*{Abbreviation}
  \cftaddtitleline{toc}{chapter}{Abbreviation}{}
}
\begin{table}[h]
\setlength{\tabcolsep}{8pt}
\renewcommand{\arraystretch}{2}
\begin{tabular}{l p{12cm}}
x86 & Yleisnimitys Intelin 1987 julkaisemalle CISC-prosessorikäskykannalle. Työssä x86:lla viitataan myös 80386:ssa käytössä olleeseen ja myöhempiin 32-bittisiin versioihin kyseisestä käskykannasta.\\
PS/2 & Sarjaväyläinen, alunperin IBM PS/2 koneessa ollut liitäntäprotokolla, joka yleistyi PC-koneiden standardiliitännäksi näppäimistölle ja hiirelle ennen USB-protokollaa.\\
RS232 & Alunperin v. 1962 esitelty asynkroninen sarjaliitäntäprotokolla, joka oli ennen USB-liitännän yleistymistä yleisin PC-koneiden oheislaitteiden liitäntäväylä.\\
USB & Universal Serial BUS, vuonna 1996 esitelty sarjaväyläinen liitäntä oheislaitteiden liitäntään.\\
IDE/PATA & Integrated Drive Electronics (Parallel AT Attachments), kiintolevyjen ja optisten asemien liittämiseen tarkoitettu 16-bittinen rinnakkainen liitäntäväylä vuodelta 1986.\\
CAN & Controller Area Network, vikasietoinen, differentiaalinen kaksijohtoinen automaatioväylä, jossa liikenne lähetetään priorisoituina sanomina vuodelta 1986\\
Apteeraus & Puun rungon jako eri puutavaralajeiksi mitta- ja laatuvaatimukset huomioiden\\
%
%OMG & Oh my god\\
%WTF & What the F\\
%TL;DR & Too long, didn't read\\
\end{tabular}
\end{table}

\newpage

%page number always on top right; also for chapter "title" page
\pagestyle{plain}
\makeevenhead{plain}{}{}{\thepage}
\makeoddhead{plain}{}{}{\thepage}

\setcounter{page}{1} %page 1 should be Introduction
\ClearWallPaper
%----------------------------------------------------------------------------------------
%	CONTENT
%----------------------------------------------------------------------------------------

\chapter{Johdanto}

Tietokoneiden käyttöikä on varsinkin vaativissa kohteissa rajallinen. Kun järjestelmät vanhenevat, niin huollon tarve lisääntyy, mutta varaosien saatavuus vähenee. Viimein voi tulla eteen piste, jossa ainoa mahdollinen vaihtoehto on korvata vanha järjestelmä uudella. Tämä saattaa vaatia suuriakin muutostöitä myös järjestelmää käyttäviin laitteisiin ja usein on järkevämpää vaihtaa koko kohde kuin tehdä siihen suuria muutostöitä.

Insinöörityön kohteena on vuonna 1998 valmistetun hakkuukoneen ajoneuvotietokoneen päivitysmahdollisuuksien tutkiminen. Alkuperäisessä, myös vuodelta 1998 olevassa ajoneuvotietokoneessa on alkanut näkyä ikääntymisen merkkejä. Laitteen komponentit hajoavat yksitellen ja se ylikuumenee etenkin kesäisin siten, että käyttöaika on vain pari tuntia kerrallaan. Osien saatavuus näin vanhaan tietokoneeseen alkaa luonnollisesti olemaan hankalaa.

Insinöörityö pyrkii selvittämään hakkuukoneen ja ajoneuvotietokoneen väliset liitännät sekä ajoneuvotietokoneen ohjelmistot ja etsimään mahdolliset korvaavat ratkaisut. Tarkoituksena on, että korvaava järjestelmä vaatisi mahdollisimman vähän muutoksia hakkuukoneeseen sekä olisi hakkuukoneen kuljettajalle toiminnallisuudeltaan vastaava kuin alkuperäinen järjestelmä. 

Haasteita työlle asettavat huollettavan hakkuukoneen sijainti yli 200 kilometrin päässä sekä alkuperäisten ohjelmien mahdollisesti asettamat vaatimukset. Lisäksi on huomioitava hakkuukoneen käyttöympäristö, joka asettaa ajoneuvotietokoneelle erityisiä vaatimuksia pölyn siedon, lämpötilan muutosten sekä tärinän keston suhteen.

\newpage

\chapter{Valmet 901 II -hakkuukone}

Hakkuukoneet eli motot (monitoimikoneet) ovat metsätraktoreita, joiden tehtävänä on hakkuun kaikki työvaiheet. Hakkuukoneet sisältävät tietokoneohjatut mittalaitteet, joilla puiden katkonta ja mittaus saadaan hoidettua tarkasti.

Työn kohteena oleva hakkuukone on nelipyöräinen Valmet 901 II vuosimallia 1998. Kone on otettu käyttöön joulukuussa 1998. Värityksen on Valmetin vanhempi punamusta-väritys. Valmet 901 on Valmetin sen aikaisen malliston pienin hakkuukone, joka on tarkoitettu ensisijaisesti harvennushakkuisiin. Alkuperäinen hakkuupää on jälkikäteen vaihdettu toiseksi, jolloin myös apteeraukseen käytettävä ohjelmisto vaihtui. Lisäksi hakkuukoneen alkuperäinen teleskooppipuomisto on vaihdettu teleskooppipuomistoon.
\newline

\begin{figure}[H]
\centering
\includegraphics[width=0.800\textwidth]{/home/th/repos/motonkone/pictures/moto_2.jpg}
\caption{Valmet 901 II}
\end{figure}

\section{Motomit-mittalaite}

Työn kohteena olevaan hakkuukoneeseen on jälkiasennettu Motomit-IT
-mittalaite, joka on korvannut hakkuukoneen alkuperäiset mittalaitteet
ja ohjelmiston. Motomit IT tukee StanForD-standardin mukaista
apteerausohjeiden tiedonsiirtoa. Motomit IT hoitaa sisäisen
kommunikaation CAN-väylää pitkin. Kommunikaatiossa alkuperäisen
ajoneuvotietokoneen kanssa käytetään RS232-väylää ja MotomitPC
-ohjelmistoa. (P. L. Oy 2008)
\newline

\begin{figure}[H]
\centering
\includegraphics[width=0.800\textwidth]{/home/th/repos/motonkone/pictures/motomit_kaavio.png}
\caption{Motomit IT:n moduulikaavio}
\end{figure}
\newpage

Motomit PC on Motomit IT:n Windows-pohjainen näyttö- ja hallintaohjelmisto Motomit IT-järjestelmälle. Motomit PC kommunikoi Motomit IT:n MCI-moduulin (Mitron CAN Interface) kanssa RS-232 -sarjaväylän avulla.
 \cite{motomit:esite}.
\newline
\begin{figure}[H]
\centering
\includegraphics[width=1.00\hsize]{/home/th/repos/motonkone/pictures/motomit_original.jpg}
\caption{Motomit PC Nero-ajoneuvopc:n ruudulla}
\end{figure}

\section{Ajoneuvo-PC Sunit Nero / Valmet Maxi}

Hakkuukoneessa käytetty tietokone on Sunitin valmistama Sunit Nero (myös nimellä Valmet MAXI)-ajoneuvotietokone. Nero on paketoitu näyttöineen ja keskusyksikköineen yhdeksi all-in-one -tyyliseksi iskunkestäväksi paketiksi. Vaikka kyseessä ei olekaan sama yksilö mikä on alunperin ollut kiinni kyseisessä hakkuukoneeessa, on se samanlainen kuin alkuperäinen ja samaa ikäluokkaa (noin 17 vuotta, vuodelta 1998). 

\begin{figure}[H]
\centering
\includegraphics[width=1.000\hsize]{/home/th/repos/motonkone/pictures/valmet_maxi.jpg}
\caption{Sunit Nero / Valmet Maxi}
\end{figure}

Emolevyn northbridge-ohjaimena on SiS 5571-ohjain. Tässä Nero-yksilössä prosessorina on AMD:n 300 MHz Socket 7-kantainen K6-prosessori vuodelta 1998 (alkuperäinen prosessori on vaihdettu useita vuosia sitten tehdyn huollon yhteydessä toiseen), 64 Mb keskusmuistia. Tietokoneen kiintolevynä oli tutkintahetkellä Western Digitalin 120 Gb:n Scorpio Blue-kiintolevy, joka sekin on vaihdettu kyseiseen yksilöön useaan kertaan. Nero-tietokoneesta löytyy 3,5" levykseasema, sekä cd-asema. Näyttöpaneelissa on 12" LCD-paneeli 800x600 pikselin resoluutiolla. Kuvasuhde on 4:3. Lisäksi koneessa on NiMH-akku BIOSin asetusten säilyttämiseen. Tietokoneesta löytyy liitännät sekä 12V että 24V jännitteille.
\newline\newline

\begin{figure}[H]
\centering
\includegraphics[width=0.500\hsize]{/home/th/repos/motonkone/pictures/processor.jpg}
\caption{AMD K6 66 MHz}
\end{figure}

Sunitin huolto on ilmoittanut, etteivät he suostu/pysty enää huoltamaan näin vanhaa tietokonetta, mikä on myös osasyynä korvaavien vaihtoehtojen kartoittamiseen.

Sunit Nerosta löytyy 3+1 sarjaporttiliitäntää (3x 9pin standardi RS-232, sekä yksi yhdistetty liitin), rinnakkaisporttiliitin (25pin Centronics), Sunitin oma BUS-liitin PS/2 hiirelle ja näppäimistölle, yhdistetty virta/GPRS/GPS -liitin sekä Data I/O-liitin jonka tarkoituksesta ei ole tietoa. Käytössä on kolme sarjaporttia, sekä virtaliitin. Sarjaporteista yksi on alkuperäiselle Terman-ohjaukselle, yksi Motomit IT:lle ja kolmas on luultavasti käytöstä poistettu GPS/GSM-moduuli.
\newline\newline

\begin{figure}[H]
\centering
\includegraphics[width=1.00\hsize]{/home/th/repos/motonkone/pictures/sunit_nero_liitannat.jpg}
\caption{Sunit Neron liitännät}
\end{figure}

\section{Valmet Terman}
Terman on Valmetin alkuperäinen DOS-pohjainen hallinta- ja apteerausohjelmisto käytettävälle hakkuukoneelle. Ohjelmiston versio on riippuvainen käytettävästä hakkuukoneesta. Kommunikointi hakkukoneen kanssa tapahtuu 9600 bitin nopeudella RS-232-sarjaporttia käyttäen. Koska mittauslaite on vaihdettu Motomit IT-järjestelmälle, käytetään Termania lähinnä hakkuukoneen ajo- ja moottoriasetusten hallintaan.
\newline\newline

\begin{figure}[H]
\centering
\includegraphics[width=1.00\hsize]{/home/th/repos/motonkone/pictures/terman_original.jpg}
\caption{Terman Nero-ajoneuvopc:n ruudulla}
\end{figure}

\newpage
\chapter{Suunnittelu}

Loppukäyttäjän (hakkuukoneen kuljettajan) toiveena on, että tietokoneen vaihtuessa käytössä olevat Terman- ja Motomit PC-ohjelmistot toimivat uudessa koneessa samaan tapaan kuin vanhassa ja näkyvät samalla tavalla kuin ohjelmistot aiemminkin ovat näkyneet. Loppukäyttäjän toiveiden lisäksi on huomioitava tietokoneen käyttöolosuhteiden asettamat vaatimukset sekä aiheeseen liittyvä lainsäädäntö ja standardit.

\section{Lainsäädäntö ja standardit}
Lain ja olosuhteiden asettamia vaatimuksia ajoneuvotietokoneelle on käsitelty seuraavissa direktiiveissä ja standardeissa:


\subsubsection{EU-direktiivi 2004/104/EY}

Direktiivi 2004/104/EY (Autoteollisuuden EMC-direktiivi) määrittää, että
1.7.2006 alkaen valmistettujen ajoneuvojen ja kiinteästi asennetun
ajoneuvoelektroniikan aiheuttamat säteilypäästöt ja päästöjen sietokyky
mitataan kyseisen direktiivin mukaisesti. Direktiiviin on julkaistu
lisäys 2005/83/EY, joka tarkentaa direktiiviä. Uusi direktiivi korvaa
aiemman direktiivin 95/54/EY.

Uusi direktiivi vaatii tyyppihyväksynnän vain laitteilta, joilla on
vaikutusta ajoneuvon hallintaan, kuljettajan asennon muuttamiseen tai
kuljettajan näkyvyysalueeseen. Laitteiden, joiden ei tarvitse olla
tyyppihyväksyttyjä, pitää täyttää kuitenkin EMC-direktiivin 89/336/ETY
tai radio- ja telepäätelaitedirektiivin 1999/5/EY vaatimukset. (S. Oy
2006) \cite{1999/5/EY} \cite{89/336/ETY}

\subsubsection{EMC-direktiivi 89/336/ETY ja
2004/108/EY}

EMC-direktiivi 89/336/ETY määrittelee ainoastaan laitteistolta
vaadittavat ominaisuudet sähkömagneettisen yhteensopivuuden
takaamiseksi. Direktiivin tarkoitus on ohjeistaa valmistajia tekemään
elektromagneettisesti yhteensopivia laitteita. Direktiivi koskee kaikkia
sähkölaitteita ja -asennuksia, joita ei direktiivissä ole erikseen
rajattu sen ulkopuolelle \cite{89/336/ETY}. Direktiivi 2004/108/EY
kumosi vanhemman direktiivin 89/336/ETY 20.7.2004 alkaen. 2004/108/EY
mm. erotteli kiinteille asennuksille ja laitteille tehtävät asennukset,
sekä yksinkertaisti vaatimustenmukaisuuden arviointimenettelyä.
\cite{2004/108/EY}

\subsubsection{Radio- ja telepäätelaitedirektiivi
1999/5/EY}

Radio- ja telepäätelaitedirektiivi 1999/5/EY määrittää radio- ja
telepäätelaitteiden yhteensopivuuden Euroopan laajuisesti. Kaikkiin
direktiivin piiriin kuuluvien laitteiden tulee olla turvallisia
käyttäjälle ja muille henkilöille sekä täyttää vaaditut
suojavaatimukset sähkömagneettisen yhteensopivuuden osalta. Lisäksi
direktiivi määrittää, että laitteistojen tulee olla rakennettuja siten,
että ne käyttävät tehokaasti radioviestintään varattua spektriä ja
resursseja. Tietyille laiteluokille on lisäksi määritelty vielä muita
vaadittuja lisäominaisuuksia kuten yksityisyyden suojan takaaminen,
yhteensopivuus muiden laitteistojen välillä, petoksia
ehkäisevien ominaisuuksien sisältäminen, tuki hätäpalveluihin pääsyn takaaville
ominaisuuksille ja/tai ominaisuudet, joilla laitteistojen
käyttö tehdään helpommaksi vammaisille {[}@1999/5/EY{]}. Direktiivi
1999/5/EY on kumottu 13.6.2016 alkaen direktiivillä 2014/53/EU
radiolaitteiden asettamista saataville markkinoilla koskevan
jäsenvaltioiden lainsäädännön yhdenmukaistamisesta. {[}@2014/53/EU{]}

\subsubsection{IP-suojaluokitus}

IP-suojaluokitus on standardissa IEC 60529 määritetty järjestelmä sähkölaitteiden tiiveyden määrittämiseksi. IP- luokitus kertoo laitteiden suojauksen pölyä ja vettä vastaan. \cite{IEC60529}

\subsubsection{IP54}

IP54-suojaluokitetut tuotteet ovat pölysuojattuja (ei täydellistä tiiveyttä, mutta ei pölykertymiä), sekä roiskesuojattuja.

\subsubsection{IP67/66}

IP67/66 -suojaluokitetut tuotteet ovat täysin pölytiiviitä ja kestävät suurella paineella tulevan vesiruiskun. IP67/66-tuotteet kestävät tärinää ja iskuja 5M3-vaatimusten mukaisesti. (DIN EN 60721-3-5, MIL-STD 810F.)

\section{Mahdolliset toteuttamisvaihtoehdot ohjelmistojen siirrolle}

\subsection{Natiivi ympäristö}

Natiivin ympäristön vaihtoehdossa ohjelmistoja käytetään ympäristössä, johon ne on aikoinaan suunniteltu toimivaksi. Hyvänä puolena tässä vaihtoehdossa on se että ohjelmat toimivat varmasti siten kun ne on suunniteltu, mutta koska käytettävät ohjelmistot ovat vanhoja, DOS- ja Windows 98- aikaisia, niin natiivin ympärsitön käyttäminen vaatisi suunnilleen samaa ikäluokkaa olevan laitteiston käyttämisen, mitä alkuperäinenkin tietokone on. Uudemmat laitteistot eivät ole enää yhteensopivia Windows 98SE / ME:n kanssa ja lisäksi Windows 98:lla on maksimirajoituksia käytettävälle laitteistolle, kuten maksimi luotettavasti käytettävä määrä keskusmuistia on 1024 Mb ja kiintolevyosiolle on rajattu maksimikoko 127,5 Gb \cite{win98:maxspecs}. Myöskään uudemmille laitteille ei löydy välttämättä toimivia ajureita, joka rajoittaa entisestään mahdollisia laitteistovaihtoehtoja. Koska ajoneuvotietokone ei ole yhdistettynä Internetiin ja koneelle on rajattu pääsy, joten tässä käyttöympäristössä Windows 98:n paikkaamattomilla haavoittuvuuksilla (Windows 98: 84 kpl, Windows 98SE 61 kpl \cite{win98:vulns}) ei ole erityistä merkitystä käytettävyyteen.

\subsection{Ohjelmistojen rajapintojen
yhteensopivuuskerros}

Ohjelmistojen rajapintojen yhteensopivuuskerros-vaihtoehdossa  käytetään käyttöjärjestelmän ja ohjelman välissä sopivia rajapintakerroksia, jolloin saadaan käyttöjärjestelmän kanssa yhteensopimattomat ohjelmat toimimaan keskenään. Vaihtoehto vaatii yhteensopivan laitteistoarkkitehtuurin alkuperäisen järjestelmän kanssa (x86). Nykyiset Windows-versiot (Windows XP+) sisältävät jo valmiiksi yhteensopivuustilan, joka mahdollistaa vanhempien ohjelmien käyttämisen uudemmissa käyttöjärjestelmissä. Linuxissa Wine -rajapinta mahdollistaa kaikenikäisten Windows-sovellusten ajamisen Linuxissa. Myös DOS:lle löytyy rajapintakerrostoteutus Linuxiin.

\subsection{Virtualisointi}

Vaihtoehdossa alkuperäisiä ohjelmistoja+käyttöjärjestelmää ajetaan virtuaalikoneessa toisen käyttöjärjestelmän päällä. Näin saavutetaan varmin yhteensopivuus ohjelmistotasolla. Oheislaitteiden siirtämisessä virtualisoidun koneen käyttöön on rajoituksia, jotka pitää huomioida virtualisointiohjelmistoja valittaessa. Vaihtoehto kuluttaa muistia enemmän ja on hieman hitaampi kuin natiivi toteutus, hyötysuhteen ollessa \textasciitilde{}90\%  natiivista \cite{virtnat_anadtech}. 

\subsection{Järjestelmäemulointi}

Vaihtoehdossa alkuperäisiä ohjelmistoja+käyttöjärjestelmä ajetaan emulaattorissa toisen järjestelmän päällä. Emuloimalla saavutetaan laitteistoarkkitehtuuririippumattomuus isäntäkoneen ja emuloitavan järjestelmän välillä. Emuloinnin haittapuolena on hitaus. Nyrkkisääntönä on 20\% hyötysuhde \cite{tinycc}, parhaat emulaattorit pääsevät n. 40-80\% hyötysuhteeseen \cite{40pperf}. Esimerksiksi Eltechs tarjoaa ExaGear Desktop-ohjelmistoaan, jolla voi ajaa x86-ohjelmia ARMv7 -ympäristössä. \cite{eltechs:exagear}. 

\subsection{Ohjelmistojen emulointi}

Vaihtoehdossa emuloidaan vain ohjelmat koko pc:n sijasta. Tämä onnistuu tietyillä ohjelmilla tiettyjen ohjelmistoarkkitehtuurien välillä \cite{tinycc},\cite{qemu_use}. Vaihtoehdolla voi ajaa mm. x86-ohjelmia ARM-prosessoreilla.

\newpage
\chapter{Toteutus}

Johtuen saatavissa olevista laitteistoista, sekä alkuperäisten ohjelmistojen arkkitehtuurista, päädyin valitsemaan testaukseen X86-arkkitehtuurilla ja Linux-käyttöjärjestelmällä toteutettavan kokonaisuuden, jossa on asennettuna sekä Winen, Dosboxin ja Dosemun päälle tarvittavat Motomit- ja Terman-ohjelmat. Lisäksi asensin virtuaalikoneeseen (Virtualbox) alkuperäisen Windows 98-käyttöjärjestelmän ja alkuperäiset ohjelmat kokeiltavaksi. Näin saamme kokeiltua Linuxin, sekä virtualisoinnin sopivuuden varsinaisen uuden ajoneuvotietokoneen mahdollisiksi toteutustavoiksi.

Linux valittiin kohdekäyttöjärjestelmäksi, koska Windows tiedetään Motomitin osalta toimivaksi vaihtoehdoksi (Motomitistä löytyy Windows 7-yhteensopiva versio). DOS-ohjelmia ei voi ajaa suoraan Windowsin päällä, koska vaikka myös uudemmista 32-bittisistä versioista (XP+) Windowseista löytyy \"Virtual DOS machine\", niin se ei rajoita ohjelmien ajonopeutta, eikä tue suorayhteyttä sarjaväylälle jonka takia tarvitaan lähes kaikissa toteustavoissa jokin lisäohjelma jotta sarjaportti saadaan toimimaan DOS:n puolella (ref: Virtual Dos Machine). Lisäksi Dosboxista löytyy sekä Windows- että Linux-versiot, jonka takia yhteensopivuuden pystyy testaamaan alustavasti Linuxin avulla. Päätökseen valita Linux vaikutti myös vahvasti oman osaamisen ja mieltymysten painottuminen Linux-puolelle, enkä nähnyt erityistä estettä olla kokeilematta järjestelmän toimivuutta Linux-käyttöjärjestelmässä.

Muita mahdollisia vaihtoehtoja toteutukselle olisi ollut X86 -pohjainen Windows 7 tai Windows 8 -käyttöjärjestelmällä oleva kannettava. Tätä ratkaisua ei käytetty, johtuen osittain käytettävästä testauskoneesta, osittain siitä että haluttiin kartoittaa käytettävien ohjelmien toiminta myös Linux-puolella, sekä osittain puuttuvan Windows-lisenssin takia.

ARMv7-pohjaisista kehitysalustoista/tietokoneista, kuten Raspberry Pi 2, sekä BeagleBone Black, olisi voinut yhdessä Eltechs ExaGear Desktop x86-tulkkausohjelmiston kanssa rakentaa kevyen ajoneuvotietokoneen, mutta tämä vaihtoehto ei ollut vielä insinöörityön selvitysvaiheessa edes tiedossa, joten koska itse tehty järjestelmä olisi lisäksi vaatinut huomattavasti enemmän suunnittelua, rakentelua ja testausta, todettiin ratkaisu liian työlääksi ja epävarmaksi.

\section{Tutustuminen Nero-ajoneuvotietokoneeseen}
\subsection{Ensimmäinen huolto 05/2004}
Ensimmäinen tutustuminen Sunitin Nero-ajoneuvotietokoneeseen tuli toukokuussa 2014 kun työn kohteena olevasta yksilöstä oli kiintolevy hajonnut. Tällöin tutkin ja selvitin tietokoneen ominaisuudet, liitännät ja ohjelmistot. Koska ohjelmistot oli saatavana kopioina Motomit-jälleenmyyjältä ja Valmetilta, ei alkuperäistä kiintolevyn sisältöä tarvinnut saada talteen. Tietokoneeseen hankittiin käytetty Hitachin 40Gb 2,5" IDE-kiintolevy, asennettiin käyttöjärjestelmä (Windows 98) ja ohjelmistot valmiiksi virtuaalikoneessa, varmuuskopioitiin asennus, sekä lopuksi 1.7.2014 testattiin tuotantoympäristössä että kaikki toimii oikein.

\subsection{Toinen huolto 09/2014}
Ensimmäisessä huollossa vaihdettu kiintolevy oli hajonnut vain muutaman kuukauden käytön jälkeen. Koska kiintolevyn koolla ja nopeudella ei ole kyseisessä tietokoneessa erityisemmin väliä, valittiin uudeksi kiintolevykorvikkeeksi paremmin tärinää ja vaihtelevia lämpö-olosuhteita kestävä SD-IDE-muistikorttiadapteri ja 2 Gb:n muistikortti. Alkuperäinen SecureDigital-standardi mahdollistaa vain 2Gb:n kapasiteetin ja väylän maksiminopeus on 12,5 Mb/s (kirjoitusnopeuden ollessa huomattavasti vähemmän, n. 1-2 Mb/s )\cite{sd:2gb}.

\subsection{Kolmas huolto 04/2015}
Huhtikuussa 2015 Nero-ajoneuvotietokoneesta hajosi BIOS:n CMOS-asetuksia ylläpitänyt NiMH-akku, jonka seurauksena ajoneuvotietokone kadottaa asetuksensa aina päävirtojen katkaisun yhteydessä. Kiintolevy pitää erikseen etsiä jokaisen päävirran katkaisun jälkeen BIOS:sta, joten tämä tekee tietokoneen käytön hankalaksi tai jopa mahdottomaksi hakkuukoneen kuljettajalle. Akkupaketin NiMH-kennot uusittiin ja BIOS-asetukset määritettiin uudelleen, jolloin kone saatiin taas takaisin käyttöön.
Testaukseen päädyttiin käyttämään käyttöjärjestelmänä Linux-distribuutiota Xubuntu 14.04. Alkuperäiset ohjelmistot Motomit PC ja Terman sovitettiin käyttöön Wine-rajapinnan ja Dosbox/Dosemu DOS -emulaattorin avulla. Järjestelmä asennettiin testausta varten vanhaan Fujitsu-Siemensin kannettavaan. Johtuen rankoista olosuhteita, tavallista kannettavaa käytetään vain käyttöjärjestelmän ja ohjelmien testaamiseksi yhdessä Moton kanssa.

\section{Käytetyt laitteistot}
\subsection{Tietokone}
Testaukseen käytetään vanhaa Fujitsu-Siemensin Amilo Pa 2510-kannettavaa vuodelta 2007. Amilo Pa 2510:ssa on AMD Turion 64 X2 TL-50 1.6 GHz prosessori, ATI Radeon Xpress X1200 128 Mb näytönohjain, 2 Gb keskusmuistia, 15.4" näyttö, sekä 120 Gb:n kiintolevy. Nämä speksit riittävät hyvin ohjelmistojen testaukseen sekä virtuaalikoneen kanssa, että emuloituna+ohjelmistorajapinnan kautta \cite{fs_amilo:review}. Puutteena kannettavassa on mm. sarjaporttien puute, sekä hajonnut näppäimistö. 

Sarjaporttiliittimet ovat poistuneet kannettavista USB:n yleistyttyä jo 2000-luvun alussa, vaikkakin vielä on myös myynnissä yrityskäyttöön kannettavia, joista löytyy vähintään yksi natiivi sarjaporttiliitin\cite{hp:laptop}. Sarjaportin puuttuminen korvataan tässä tapauksessa USB-sarjaporttiadapterilla, jonka avulla voidaan kyseistä kannettavaa tietokonetta käyttää testaukseen.

Rikkinäinen näppäimistö korvataan erillisellä USB-väylään liitettävällä näppäimistöllä testauksen ajaksi. Sähkönsyöttö testauksen ajan hoituu kannettavan omalla akulla, jossa on vielä kapasiteettia pitää kannettava käynnissä testauksen ajan.

\subsection{USB-Sarjaportti -adapteri}
Käytettävä USB-sarjaporttiadapteri on eBaystä testaukseen hankittu 2-porttinen Unitek USB to Dual Serial Cable Y-106 \cite{serial:unitek}. Y-106 -adapteri käyttää MosChip Semiconductorin MCS7820 -piiriä muunnokseen. Adapteri valittiin, koska siinä oli tarvittavat 2 sarjaporttia, adpaterin myyntihinta oli edullinen (n. 25€ sisältäen toimituksen), sekä käytetylle muunninpiirille löytyi ajurituki sekä Windows-, että Linux -käyttöjärjestelmille.


\section{Käytetyt ohjelmat}
\subsection{Xubuntu}

Xubuntu 14.04 on Ubuntu 14.04:aan perustuva (joka taas on Debianiin perustuva) Linux-distro, jossa 
käyttöliittymäksi on vaihdettu XFCE 4.10 ja mukana toimitettavista ohjelmistoista on valittu kevyemmät 
versiot kuin Ubuntussa.\cite{xubuntu:about}. Xubuntu valittiin käyttöjärjestelmäksi, koska testaamiseen 
käytettävä kannettava on hieman vanhempi ja Xubuntu on todettu jo aiemmin toimivaksi käyttöjärjestelmäksi ko. kannettavassa.

Käyttöjärjestelmäksi valittiin Xubuntusta pitkän tuen versio 14.04.
Xubuntu asennettiin oletusasetuksilla testikoneeseen. Lisäksi Xubuntu
laitettiin kirjautumaan sisään automaattisesti, sekä Dosbox ja MotomitPC
laitettiin käynnistymään automaattisesti sisäänkirjautumisen yhteydessä.


\subsection{Wine}

Wine on Windows-yhteensopivuuskerros Unixin kaltaisiin (mm. Unix/Linux/OS X/Solaris) käyttöjärjestelmiin, joka mahdollistaa Windows-ohjelmien käyttämisen käytetyssä käyttöjärjestelmässä. Wineä tarvitaan, että apteeraukseen käyettävä Motomit PC-ohjelmisto saadaan toimimaan Linuxissa. Vaikka Motomit sisältääkin taustajärjestelmänään Linuxin\cite{motomit:manual}, niin käyttöliittymäänä tomivasta Motomit PC -ohjelmasta on vain olemassa Windows-versio.

\subsection{Dosbox}

Dosbox on DOS-emulaattori, joka emuloi IBM PC-yhteensopivaa tietokonetta 286/386-prosessorilla, sekä monia kyseisen aikakauden laitteistoja. Dosbox sisältää lisäksi suoratuen sarjaportille, niin se on valittu käytettäväksi DOS-emulaattoriksi. Dosboxia tarvitaan, että Terman-ohjelmisto saadaan toimimaan uudemmissa Windowseissa ja Linuxeissa.

\subsection{Dosemu}
Dosemu on virtuaalikone / DOS-rajapintakerros Linuxille, joka mahdollistaa DOS-ohjelmien ajamisen x86-arkkitehtuuria käyttävässä Linux-käyttöjärjestelmässä. Dosemu valittiin myös testattavaksi vaihtoehdoksi Termanin saamiseksi toimimaan Linux-käyttöjärjestelmässä.

\section{Asennus ja asetusten säätö}
Käytetylle USB-sarjaporttiadapterille lisättiin oma udev-sääntö, jonka ansiosta adapterin kaksi sarjaporttia tulevat näkyviin Linuxissa samoilla laitetiedosto (Device file)-nimillä, vaikka tietokoneeseen olisi kytkettynä useampi erillainen USB-sarjaporttiadapteri.

Jotta sarjaportit saa toimimaan Winessä Windowsin käyttämillä COM-porttinimillä Windows-ohjelmien puolella, tulee laitetiedostoista tehdä symboliset linkit \textasciitilde{}/.wine/dosdevices/ -hakemistoon halutuilla nimillä (COM1 ja COM2).\cite[s. 21]{wine:manual}

Dosboxin ja Dosemun asetuksiin tuli määrittää käytetävät sarjaportit. Lisäksi Dosboxista muutettiin suoritusnopeutta ja ruudun kokoa suuremmaksi, jotta käytettävyys olisi parempi \cite{dosbox:conf}. Molemmat ohjelmat jätettiin ikkunointimoodiin, jotta yhteiskäyttö Motomit PC:n kanssa olisi helpompaa.

\section{Testaus}
Kootun Linux-järjestelmän testaus suoritettiin metsässä hakkuukoneen luona. Rajoittavaksi tekijäksi asettui testikannettavan akun kesto, joka oli vain n. 15 minuuttia. Tässä ajassa saatiin kuitenkin testattua alustavasti Motomit PC- ja Terman-ohjelmistojen toimivuus Winen ja Dosbox/Dosemun avulla suoritettuna. Valitettavasti Virtuaalikoneeseen asennettuja ohjelmia ei ehditty testaamaan ennen kuin akku loppui.

\begin{figure}[H]
\centering
\includegraphics[width=1.00\hsize]{/home/th/repos/motonkone/pictures/motomit_linux.png}
\caption{Motomit PC-ohjelmisto Winessä ajettuna.}
\end{figure}

\newpage
\chapter{Yhteenveto}\
Insinöörityön tavoitteena oli selvittää työn kohteena olevan 17v vanhan hakkuukoneen alkuperäisen ajoneuvotietokoneen liitännät ja ohjelmistot, sekä selvittää mahdolliset vaihtoehdot korvata vanha, käyttöikänsä loppupäässä oleva alkuperäinen ajoneuvotietokone uudemmalla, hakkuukoneen järjestelmien kanssa yhteensopivalla vaihtoehdolla.

Alkuperäistä ajoneuvotietokonetta tutkimalla, sekä mahdollisia vaihtoehtoja punnitsemalla päädyttiin testaamaan alkuperäisten ohjelmistojen toimivuutta yhteensopivuusrajapintojen kautta tavallisessa, useamman vuoden vanhassa Linux-käyttöjärjestelmällä varustetussa kannettavassa tietokoneessa. Testaamalla selvisi, että käytettävät alkuperäiset ohjelmistot toimivat USB-sarjaporttiadapterin ja Linuxin kautta suoritettuina. Tästä voidaan päätellä että alkuperäinen Sunit Nero-ajoneuvotietokone on helpostikin korvattavissa uudemmalla olosuhteita kestävällä ajoneuvotietokoneella tai kannettavalla.

Vaatimuksina uudelle tietokoneelle on lähinnä tarvittava olosuhteiden kesto, sekä tarvittavat liittimet. Hakkuukoneen järjestelmät vaativat kaksi sarjaporttiliitäntää, joilla kommunikointi hakkuukoneen ja ajoneuvotietokoneen välillä tapahtuu. Hakkuukoneen sähköjärjestelmä on 24V, joka tulee ottaa huomioon tietokoneen sähkönsyöttöä miettiessä. 

Suurimmat haasteet insinöörityön kanssa liittyivät pääsyyn työn kohteena olevan hakkuukoneelle. Hakkuukone oli aktiivisessa käytössä Oriveden metsissä koko insinöörityön ajan ja vierailut koneelle tuli lopuksi järjestettyä alkuperäisen ajoneuvotietokoneen huoltojen yhteydessä. Hakkuukoneen kommunikointi ajoneuvotietokoneen kanssa osoittautui kuitenkin yksinkertaiseksi, joten järjestyneet käyntikerrat riittivät hyvin selvittämään alkuperäisen järjestelmän toimintaa sekä testaamaan mahdollisessa uudessa tietokoneessa käytettäviä ratkaisuja. Insinöörityöstä jäi puutumaan alustava analyysi Terman- ja Motomit PC-ohjelmistojen käyttämille protokollille, koska en testikannettavan huonon akunkeston takia saanut tallennettua näytettä sarjaporttien liikenteestä. Tämän liikenteen avulla olisi voinut myös simuloida hakkuukonetta syöttämällä kaapatun datan mikrokontrollerin tai toisen tietokoneen avulla testikannettavaan.

Insinöörityössä saavutettiin asetetut tavoitteet ja saatiin pohjustettua uuden tietokoneen vaatimuksia ja vaihtoehtoja. Suunnitelmissa on, että hakkuukoneeseen saadaan hankittua ja rakennettua uusi ajoneuvotietokone varajärjestelmäksi, joka voidaan ottaa käyttöön alkuperäisen ajoneuvotietokoneen hajotessa seuraavan kerran.

Lopuksi haluaisin kiittää rakasta vaimoani, joka oikoluki insinöörityön julkaistavaan kuntoon, Tuukka Haapaniemeä toimimisena apulaisena alkuperäisen koneen huollon ja tutkimisen yhteydessä, sekä Arto Saalia, jonka hakkuukoneeseen insinöörityö kohdistui ja joka auttoi käytännön järjestelyissä.


%----------------------------------------------------------------------------------------
%   BIBLIOGRAPHY 
%----------------------------------------------------------------------------------------

\IfLanguageName{finnish}{\bibliographystyle{vancouver_fi}}{\bibliographystyle{vancouver}}
%line space
%\singlespacing %removed otherwise the appendix are also single space
\begin{flushleft}
\begin{singlespacing}
\bibliography{motonkone_biblio}
\end{singlespacing}
\end{flushleft}

%for conting the pages
\label{LastPage}~


%----------------------------------------------------------------------------------------
%   APPENDICES 
%----------------------------------------------------------------------------------------
%avoid that the last page of bib get appendix header
\clearpage
%start appendix
\appendix
%no page number for appendix in table of content
\addtocontents{toc}{\cftpagenumbersoff{chapter}}
%appendix sections and subsections not in table of content
\settocdepth{chapter}
%add "Appendices" in the table of content
\addappheadtotoc
%force smaller vertical spacing in table of content
%!!! There can be some fun depending if the appendices have (sub)sections or not :D
% You will have to play with these numbers and eventually copy the \pretocmd line on before some \chapter and force another number.
\addtocontents{toc}{\vspace{11pt}}
\pretocmd{\chapter}{\addtocontents{toc}{\protect\vspace{-24pt}}}{}{}
%have Appendix 1 (instead of Appendix A)
\renewcommand{\thechapter}{\arabic{chapter}} 

\newcommand\liite[1]{
%each appendix restart page num to one
\setcounter{page}{1}
%special counter for appendix TODO: this is a ugly quick hack :( Should find a better way to count the page per appendix.
\newtotcounter{appx#1}
%overwrite the header
\makeevenhead{plain}{}{}{\appname \thechapter \\ \thepage\,(\stepcounter{appx#1}\total{appx#1})}
\makeoddhead{plain}{}{}{\appname \thechapter \\ \thepage\,(\stepcounter{appx#1}\total{appx#1})}}

\liite{1}
\chapter{Linux configurations}\label{appx:first}


\section{Udev rules}
\begin{lstlisting}
#/etc/udev/rules.d/10-local.rules
ACTION=="add", DRIVERS=="mos7840",ATTRS{port_number}=="0", SYMLINK+="ttyMoto0"
ACTION=="add", DRIVERS=="mos7840",ATTRS{port_number}=="1", SYMLINK+="ttyMoto1"
\end{lstlisting}

\section{Dosbox configuration file}
\begin{lstlisting}
[sdl]
fullscreen=false
fulldouble=false
fullresolution=800x600
windowresolution=800x600
output=overlay
autolock=false
sensitivity=100
waitonerror=true
priority=higher,normal
mapperfile=mapper-0.74.map
usescancodes=true

[cpu]
core=auto
cputype=auto
cycles=max
cycleup=10
cycledown=20

[serial]
serial1=directserial realport:ttyMoto0
serial2=directserial realport:ttyMoto1
\end{lstlisting}

\section{Dosemu configuration file}
\begin{lstlisting}
TODO
\end{lstlisting}

\section{Wine configurations}
\begin{lstlisting}
#Create symbolic links to .wine/dosdevices/
sudo ln -s /dev/ttyMoto0 /home/user/.wine/dosdevices/com1
sudo ln -s /dev/ttyMoto1 /home/user/.wine/dosdevices/com2
\end{lstlisting}




\end{document}
