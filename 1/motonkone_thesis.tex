%----------------------------------------------------------------------------------------
%	Metropolia Thesis LaTeX Template
%----------------------------------------------------------------------------------------
% License:
% This work is licensed under the Creative Commons Attribution 4.0 International License. To view a copy of this license, visit http://creativecommons.org/licenses/by/4.0/.
%
% Authors:
% Panu Leppäniemi, Patrik Luoto and Patrick Ausderau
%
% Credits:
% Panu Leppäniemi: abstract, def, cleaning,...
% Patrik Luoto: title page, abstract in Finnish, abbreviation, math,...
% Patrick Ausderau: initial version, style, table of content, bibliography, figure, appendix, table, source code listing...
%
% Please:
% If you find mistakes, improve this template and alike, please contribute by sharing your improvements and/or send us your feedback there: https://github.com/panunu/metropolia-thesis-latex
% And of course, if you improve it, add yourself as an author.
%
% Compiler:
% Use XeLaTeX as a compiler.

%----------------------------------------------------------------------------------------
%	THESIS
%----------------------------------------------------------------------------------------

\def\thesislang{finnish} %change this depending on your language
\author{Tomi Haapaniemi}
\def\thesis{Opinnäytetyö/Thesis}
\def\alaotsikko{Alaotsikko/Subtitle}

%Finnish section
\def\otsikko{Hakkuukoneen ajoneuvotietokoneen päivitysmahdollisuudet uudempaan}
\def\tutkinto{Insinööri (AMK)}
\def\kohjelma{Tietotekniikan koulutusohjelma}
\def\suuntautumis{Ohjelmistotekniikka/Sulautettu tietotekniikka}
\def\ohjaajat{
Etunimi Sukunimi, Titteli\newline
Etunimi Sukunimi, Titteli
}
\def\avainsanat{avainsanat}
\def\pvm{\ddmmyyyydate\today}

%English section, for abstract
\title{Possibilities to upgrade embedded computer of Harvester}
\def\metropoliadegree {Bachelor of Engineering}
\def\metropoliadegreeprogramme {Degree Programme in Information Technology}
\def\metropoliaspecialisation {Software Engineering/Embedded Engineering}
\def\metropoliainstructors {
First name Last name, Title (for example: Project Manager)\newline
First name Last name, Title (for example: Principal Lecturer)
}
\def\metropoliakeywords {Keywords}
\date{\today}

%----------------------------------------------------------------------------------------
%	GLOBAL STYLES
%----------------------------------------------------------------------------------------

\documentclass[11pt,a4paper,oneside,article]{memoir}
\usepackage[\thesislang]{babel} 
\usepackage{float}
\usepackage{iflang}
\usepackage{amsmath}
\usepackage{amsfonts}
\usepackage{amssymb}
\usepackage{fontspec}
\usepackage{tocloft}
\usepackage{titlesec}
\usepackage[hyphens]{url}
\usepackage{mathtools}
\usepackage{wallpaper}
\usepackage{datetime}
\usepackage[bookmarksdepth=subsection]{hyperref} % for automagic pdf links for toc, refs, etc.
\usepackage[amssymb]{SIunits}
\usepackage[version=3]{mhchem}
\usepackage{pgfplots} %simple plots etc
\usepackage{pgfplotstable}
\usepackage{tikz} % mindmaps, flowcharts, piecharts, examples at http://www.texample.net/tikz/examples/
\usetikzlibrary{shapes.geometric, arrows}


\renewcommand{\dateseparator}{.}
%condition for adding or not space in TOC
\usepackage{etoolbox}
%for compact list
\usepackage{enumitem}
%for block comment
\usepackage{verbatim}
%for "easier" references
\usepackage{varioref}
%forcing single line spacing in bibliography
\DisemulatePackage{setspace}
\usepackage{setspace}
%including figure (image)
\usepackage{graphicx}
%change the numbering for figure
\usepackage{chngcntr}
%strike trough
\usepackage{ulem}
%euro symbol
\usepackage{eurosym}
%try to count
\usepackage{totcount}
%insert source code
\usepackage{listings}
\usepackage{caption}
\usepackage{color}
%force the width of a table instead of column
\usepackage{tabularx}
\usepackage{booktabs} %why not booktabs? :3

\newcommand\tn[1]{\textnormal{#1}} %use \tn instead of \textnormal
\newcommand\reaction[1]{\begin{equation}\ce{#1}\end{equation}} %\reaction{} for chemical reactions

%NORMAL TEXT
%all text, title, etc. in the same font: Arial
%replace with arial.ttf if you have the fontfile
%NOTE: fontname is case-sensitive
\setmainfont
[BoldFont=LiberationSans-Bold.ttf,
ItalicFont=LiberationSans-Italic.ttf,
BoldItalicFont=LiberationSans-BoldItalic.ttf]
{LiberationSans-Regular.ttf}
%line space
\linespread{1.5}
%\doublespacing
%margin
\usepackage[top=2.5cm, bottom=3cm, left=4cm, right=2cm, nofoot]{geometry}
\setlength{\parindent}{0pt} %first line of paragraph not indented
\setlength{\parskip}{16.5pt} %one empty line to separate paragraph
%list with small line space separation
\tightlists

%IMAGE - FIGURE
%the figures should be placed in the "illustration" folder
\graphicspath{{illustration/}}
%figure number without chapter (1.1, 1.2, 2.1) to (1, 2, 3)
\counterwithout{figure}{chapter}
%border around images
\setlength\fboxsep{0pt}
\setlength\fboxrule{0.5pt}
%caption font size
\captionnamefont{\small}
\captiontitlefont{\small}
%space after figure caption (and other float elements)
\setlength{\belowcaptionskip}{-7pt}

%TABLE
\counterwithout{table}{chapter}

%SOURCE CODE
\definecolor{darkgray}{rgb}{.4,.4,.4}
\definecolor{purple}{rgb}{0.65, 0.12, 0.82}
\lstset{
extendedchars=true,
captionpos=b,
caption=\footnotesize,
basicstyle=\singlespacing\ttfamily,%\small\fontfamily{"Courier"}\selectfont,
keywordstyle=\color{blue}\bfseries,
commentstyle=\color{purple}\itshape,
identifierstyle=\color{black},
stringstyle=\color{red},
showstringspaces=false,
showspaces=false,
numbers=left,
numberstyle=\footnotesize,
numbersep=9pt,
breaklines=true,
tabsize=2,
showtabs=false,
xleftmargin=1cm
}
\IfLanguageName {finnish} {\renewcommand{\lstlistingname}{Listaus}} {} % what is a good translation for this?
%\counterwithout{lstlisting}{chapter}
%moved after begin document, otherwise does not compile

%TOC
%change toc title
\IfLanguageName {finnish} {\addto{\captionsfinnish}{\renewcommand*{\contentsname}{Sisällys}}} {}
%remove dots
\renewcommand*{\cftdotsep}{\cftnodots}
%chapter title and page number not in bold
\renewcommand{\cftchapterfont}{}
\renewcommand{\cftchapterpagefont}{}
%sub section in toc
\setcounter{tocdepth}{2}
%subsection numbered
\setcounter{secnumdepth}{2}
\renewcommand{\tocheadstart}{\vspace*{-15pt}}
\renewcommand{\printtoctitle}[1]{\fontsize{13pt}{13pt}\bfseries #1}
\renewcommand{\aftertoctitle}{\vspace*{-22pt}\afterchaptertitle}
%spacing afer a chapter in toc
\preto\section{%
  \ifnum\value{section}=0\addtocontents{toc}{\vskip11pt}\fi
}
%spacing afer a section in toc
\renewcommand{\cftsectionaftersnumb}{\vspace*{-3pt}}
%spacing afer a subsection in toc
\renewcommand{\cftsubsectionaftersnumb}{\vspace*{-1pt}}
%appendix in toc with "Appendix " + num
\IfLanguageName {finnish} {
  \renewcommand*{\cftappendixname}{Liite\space}
  \renewcommand{\appendixtocname}{Liitteet}
}{\renewcommand*{\cftappendixname}{Appendix\space}}
%appendix header
\IfLanguageName {finnish} {\def\appname{Liite\space}}{\def\appname{Appendix\space}}

%TITLES
%chapter title
\titleformat{\chapter}
{\fontsize{13pt}{13pt}\bfseries\linespread{1}}
{\thechapter}{.5cm}{}
\titlespacing*{\chapter}{0pt}{.32cm}{9pt}
\titleformat{\section}
{\fontsize{12pt}{12pt}\linespread{1}}
{\thesection}{.5cm}{}
\titlespacing*{\section}{0pt}{14pt}{6pt}
\titleformat{\subsection}
{\fontsize{12pt}{12pt}\linespread{1}}
{\thesubsection}{.5cm}{}
\titlespacing*{\subsection}{0pt}{14pt}{6pt}


%QUOTE
\renewenvironment{quote}
  {\list{}{\rightmargin=0pt\leftmargin=1cm\topsep=-10pt}%
  \item\relax\fontsize{10pt}{10pt}\singlespacing}
  {\endlist}

%BIBLIOGRAPHY
%bibliography title to be "references"
\IfLanguageName {finnish} {\addto{\captionsfinnish}{\renewcommand*{\bibname}{Lähteet}}} {\renewcommand\bibname{References}}
\makeatletter %reference list option change
\renewcommand\@biblabel[1]{#1\hspace{1cm}} %from [1] to 1 with 1cm gap
\makeatother %
\setlength{\bibitemsep}{11pt}

%count the appendices (since the chapter counter is reset after \appendix).
%! require to complie 2 times
\regtotcounter{chapter}

%TITLE PAGE
\makeatletter
\renewcommand{\maketitle}{
\thispagestyle{empty}
\ThisCenterWallPaper{1}{viiva}
%
\vspace*{9.5cm}
\tn{\LARGE\@author\\[22pt]\Huge\IfLanguageName {finnish}{\otsikko}{\@title}\\[22pt]\LARGE\alaotsikko\\[1.75cm]}

\parbox{.7\linewidth}{
\IfLanguageName {finnish}{
  Metropolia Ammattikorkeakoulu\\
  \tutkinto \\
  \kohjelma \\
  \thesis\\
  \pvm
} {
  Helsinki Metropolia University of Applied Sciences\\
  \metropoliadegree \\
  \metropoliadegreeprogramme \\
  \thesis\\
  \ddmmyyyydate\today %to be checked date format? 
}
}
\ThisLRCornerWallPaper{1}{metropolia}
%
\clearpage
}
\makeatother

\makepagestyle{tiivis}
\makeevenhead{tiivis}{}{}{Tiivistelmä}
\makeoddhead{tiivis}{}{}{Tiivistelmä}

\makepagestyle{abstract}
\makeevenhead{abstract}{}{}{Abstract}
\makeoddhead{abstract}{}{}{Abstract}

\begin{document}
\counterwithout{lstlisting}{chapter}

%page number always on the top right, clear the "chapter/section" head
\pagestyle{myheadings}
\markright{}
%clear chapter "title" foot page
\makeevenfoot{plain}{}{}{}
\makeoddfoot{plain}{}{}{}

%----------------------------------------------------------------------------------------
%	TITLE PAGE
%----------------------------------------------------------------------------------------

\maketitle
\newpage
\LRCornerWallPaper{1}{footer}

%----------------------------------------------------------------------------------------
%    Tiivistelmä
%----------------------------------------------------------------------------------------

\thispagestyle{tiivis}
\begin{tabular}{ | p{4,7cm} | p{10,3cm} |}
  \hline
  Tekijä(t) \newline
  Otsikko \newline\newline 
  Sivumäärä \newline
  Aika
  & 
  \makeatletter
  \@author \newline 
  \otsikko \newline\newline 
  \makeatother
  \pageref*{LastPage} sivua + \total{chapter} liitettä \newline %! if no appendices, risk to count total of chapter :D
  \pvm		
  \\ \hline
  Tutkinto & \tutkinto
  \\ \hline
  Koulutusohjelma & \kohjelma
  \\ \hline
  Suuntautumisvaihtoehto & \suuntautumis
  \\ \hline
  Ohjaaja(t) & \ohjaajat
  \\ \hline
  \multicolumn{2}{|p{15cm}|}{\begin{singlespacing}\vspace{-22pt}
  Tämä on tiivistelmän ensimmäinen kappale. Tiivistelmän kappaleet loppuvat komentoon newline, jotta saadaan yksi tyhjä rivi aikaiseksi. \newline
  
  Tämä on tiivistlemän toinen kappale.
  \end{singlespacing}} \\[14cm] \hline
  Avainsanat & \avainsanat
  \\ \hline
\end{tabular}
\clearpage

%----------------------------------------------------------------------------------------
%	ABSTRACT
%----------------------------------------------------------------------------------------

\pagestyle{abstract}
\begin{tabular}{ | p{4,7cm} | p{10,3cm} |}
  \hline
  Author(s) \newline
  Title \newline\newline 
  Number of Pages \newline
  Date
  & 
  \makeatletter
  \@author \newline
  \@title \newline\newline
  \pageref*{LastPage} pages + \total{chapter} appendices \newline %! if no appendices, risk to count total of chapter :D
  \IfLanguageName {finnish} {\foreignlanguage{english}{\longdate\@date}} {\@date}
  \makeatother
  \\ \hline
  Degree & \metropoliadegree
  \\ \hline
  Degree Programme & \metropoliadegreeprogramme
  \\ \hline
  Specialisation option & \metropoliaspecialisation
  \\ \hline
  Instructor(s) & \metropoliainstructors
  \\ \hline
  \multicolumn{2}{|p{15cm}|}{\begin{singlespacing}\vspace{-22pt}
  Abstract content
  \end{singlespacing}} \\[14cm] \hline
  Keywords & \metropoliakeywords
  \\ \hline
\end{tabular}
\clearpage

%----------------------------------------------------------------------------------------
%	Acknowledgement ?
%----------------------------------------------------------------------------------------
%\chapter*{Acknowledgement}
%Thanks to my cat
%\clearpage

%----------------------------------------------------------------------------------------
%	TABLE OF CONTENTS
%----------------------------------------------------------------------------------------

\makeevenhead{plain}{}{}{}
\makeoddhead{plain}{}{}{}
\pagestyle{empty} %remove page number in toc (if longer than 2 pages)
\tableofcontents*
\pagestyle{empty} %remove page number in toc (if longer than 1 pages)
\clearpage
\pagestyle{plain}

%list of figure, tables comes here...


%----------------------------------------------------------------------------------------
%    Lyhenteet / Abbreviation
%----------------------------------------------------------------------------------------

\pagestyle{empty}
\setlength{\parskip}{1cm}
\IfLanguageName {finnish} {
  \chapter*{Lyhenteet}
  \cftaddtitleline{toc}{chapter}{Lyhenteet}{}
} {
  \chapter*{Abbreviation}
  \cftaddtitleline{toc}{chapter}{Abbreviation}{}
}
\begin{table}[h]
\setlength{\tabcolsep}{8pt}
\renewcommand{\arraystretch}{2}
\begin{tabular}{l p{12cm}}
OMG & Oh my god\\
WTF & What the F\\
TL;DR & Too long, didn't read\\
\end{tabular}
\end{table}

\newpage

%page number always on top right; also for chapter "title" page
\pagestyle{plain}
\makeevenhead{plain}{}{}{\thepage}
\makeoddhead{plain}{}{}{\thepage}

\setcounter{page}{1} %page 1 should be Introduction
\ClearWallPaper
%----------------------------------------------------------------------------------------
%	CONTENT
%----------------------------------------------------------------------------------------

\chapter{Johdanto}

Tietokoneiden käyttöikä on varsinkin vaativissa kohteissa rajallinen.
Kun järjestelmien ikä kasvaa niin varaosien saatavuus vähenee. Viimein
ollaan pisteessä, missä ainoa vaihtoehto on korvata vanha järjestelmä
uudella. Tämä saattaa vaatia mittavia päivitysprojekteja myös
järjestelmää tukeviin kokonaisuuksiin ja vaihdon kannattavuus suhteessa
hyötyyn on huono.

Järjestelmänä on vuonna xxx valmistetun metsätraktori, eli
tuttavallisemmin moto. Motolla on n. yy v käyttöikää jäljellä ja moton
ajoneuvotietokone, jolla hallitaan koneen moottoriasetuksia, että myös
puiden kadon hallintaa, alkaa olemaan elinikänsä loppupäässä.
Alkuperäinen tietokone lakkaa toimimasta kokonaan kuumennettuaan liikaa,
kiintolevy on hajonnyt useaan otteeseen ja akustot alkavat olemaan
uusimisen tarpeessa. Jos laitteiston vaihtaisi kokonaan uudempaan olisi
kyseessä sen verran kallis toimenpide(n. 20 000\euro{}), ettei sitä
kannata tehdä enää kyseiseen metsätraktoriin. Tämän takia tutkimmekin
vaihtoehtoisia ratkaisuita lisätä metsätraktorin käytössä olevalle
tietokonejärjestelmälle elinikää.

Insinöörityön aiheena on löytää motossa käytetylle 15v vanhan Sunit
Nero-ajoneuvotietokoneelle korvaava uudempi, mutta yhteensopiva
tietokone. Tavoitteena on uuden laitteen yhteensopivuus vanhan
kiinnitysjärjestelmän kanssa, liitinyhteensopivuus, sekä ohjelmallisen
tason yhteensopivuus. Aluksi tutustutaan käytössä olevaan laitteistoon
ja sen asettamiin vaatimuksiin. Seuraavaksi käydään läpi mahdollisia
toteuttamisvaihtoehtoja ja niiden ominaisuuksia. Asennetaan uusi
ohjelmisto testikannetavaan ja testataan järjestelmä
tuotantoympäristössä. Lopuksi tutkitaan mahdolliset

Haasteita työlle asettavat vanhat ohjelmistot, tärinää ja pölyä ja
vaihtelevia lämpötiloja sisältävä työympäristö. Metsätraktori on
huoltoja lukuunottamatta metsässä kesät talvet ja lämpötilat vaihtelevat
talvella -20 asteesta +20 asteeseen ja kesäisin lämmöt voivat nousta
ohjaamossa jopa +50-60 asteeseen.

\chapter{Taustaa}
\section{Hakkuukoneet}

Hakkuukoneet eli motot (monitoimikoneet) ovat metsätraktoreita, joiden
tehtävänä on hakkuun kaikki työvaiheet. Hakkuukoneet sisältävät
tietokoneistetu mittalaitteet joilla katkonta ja mittaus saadaan
hoidettua tarkasti.

\subsection{Hakkuukone Valmet xxx}

Lorem ipsum dolor sit amet, consectetur adipiscing elit. Nam sed gravida
ex. Sed leo nisl, viverra in efficitur eget, imperdiet vel nibh. Donec
gravida sapien facilisis nisl rhoncus, sit amet ullamcorper nulla
congue. Donec egestas nisi sed finibus tempus. Integer convallis
suscipit magna et sollicitudin. Fusce gravida nisl eros, sit amet congue
odio aliquam vel. Vivamus congue massa eget est efficitur, dapibus
lacinia velit porttitor. Nunc consectetur sit amet augue vel ultrices.
Nullam hendrerit nisi efficitur tincidunt molestie.

\subsection{Motomit-mittalaite}

Kohteena olevaan Valmet xxx-hakkuukoneeseen on jälkiasennettu Motomit-IT
-mittalaite, joka on korvannut hakkuukoneen alkuperäiset mittalaitteet
ja ohjelmiston hakkuussa. Motomit IT tukee StanForD-standardin mukaista
apteerausohjeiden tiedonsiirtoa. Motomit IT hoitaa sisäisen
kommunikaation CAN-väylää pitkin. Kommunikaatiossa alkuperäisen
ajoneuvotietokoneen kanssa käytetään RS232-väylää ja MotomitPC
-ohjelmistoa. (P. L. Oy 2008)

\begin{figure}[H]
\centering
\includegraphics[width=1.000\textwidth]{/home/th/repos/motonkone/pictures/motomit_kaavio.png}
\caption{Motomit IT:n moduulikaavio}
\end{figure}

\section{Alkuperäinen PC-järjestelmä}

\subsection{Ajoneuvo-PC Sunit Nero / Valmet Maxi}

Hakkuukoneessa kiinni oleva ajoneuvo-PC Sunit Nero / Valmet
Maxi on valmistettu {[}joskus 1997-1999{]}. Sunit Nero on oikeastaan
kannettava, johon on modifioitu ulkopuoliset liittimet ja tukevampi
runko. Kotelointi koostuu XXX-muovisista ulkokuorista, sekä
metallilevystä (teräs? alumiini?), jonka molemmin puolin on komponentit
kiinnitetty. Toisella puolen on emolevy,prosessori,NIMH-akku (asetusten
säilytystä varten? ymmärtääkseni) ja liittimet, toisella puolen
kiintolevy, levykeasema,cd-asema ja näyttö. Näyttöpaneeli on 4:3 800x60
LCD.

Ajoneuvotietokoneeessa on ollut koko käyttöiän (\textasciitilde{}15v)
erilaisia ongelmia. Alkuperäinen laite on vaihdettu syystä x vuonna y.
Nykyisestä laitteesta on kiintolevy hajonnut vuonna xxxx ja 2014,
jolloin pääsin ensimmäisen kerran tutustumaan laitteeseen paremmin.
Prosessori on vaihdettu v. zzzz.

Koneen ongelmana on ollut viime aikoina ylikuumeneminen. Kuumennettuaan
siitä tulee epävakaa eikä se lähde päälle ennekuin jäähdyttyään, joka
kuumana kesäpäivänä traktorin hytissä vie aikaa

\begin{figure}[H]
\centering
\includegraphics[width=1.000\hsize]{/home/th/repos/motonkone/pictures/valmet_maxi.jpg}
\caption{Sunit Nero / Valmet Maxi}
\end{figure}

\begin{figure}[H]
\centering
\includegraphics[width=0.500\hsize]{/home/th/repos/motonkone/pictures/processor.jpg}
\caption{AMD K6 66 MHz}
\end{figure}

\subsection{Valmet Terman}
Terman on Valmetin alkuperäinen DOS-pohjainen hallinta- ja apteerausohjelmisto käytettävälle hakkuukoneelle. Ohjelmiston versio on riippuvainen käytettävästä hakkuukoneesta. Kommunikointi hakkukoneen kanssa tapahtuu 9600 bitin nopeudella RS-232-sarjaporttia käyttäen. Koska mittauslaite on vaihdettu Motomit IT-järjestelmälle, käytetään Termania hakkuukoneen moottoriasetusten hallintaan.

\subsection{Motomit PC}
Motomit PC on Motomit IT:n Windows-pohjainen näyttö- ja hallintaohjelmisto Motomit IT-järjestelmälle. Motomit PC kommunikoi Motomit IT:n kanssa RS-232 -sarjaväylän avulla. \cite{motomit:esite}

\section{Standardeja}

\subsection{EU-direktiivi 2004/104/EY}

Direktiivi 2004/104/EY (Autoteollisuuden EMC-direktiivi) määrittää, että
1.7.2006 alkaen valmistettujen ajoneuvojen ja kiinteiästi asennetun
ajoneuvoelektroniikan aiheuttamat säteilypäästöt ja päästöjen sietokyky
mitataan kyseisen direktiivin mukaisesti. Direktiiviin on julkaistu
lisäys 2005/83/EY, joka tarkentaa direktiiviä. Uusi direktiivi korvaan
aiemman direktiivin 95/54/EY.

Uusi direktiivi vaatii tyyppihyväksynnän vain laitteilta, joilla on
vaikutusta ajoneuvon hallintaan, kuljettajan asennon muuttamiseen tai
kuljettajan näkyvyysalueeseen. Laitteiden, joiden ei tarvitse olla
tyyppihyväksyttyjä, pitää täyttää kuitenkin EMC-direktiivin 89/336/ETY
tai radio- ja telepäätelaitedirektiivin 1999/5/EY vaatimukset. (S. Oy
2006) {[}@1999/5/EY{]} {[}@89/336/ETY{]}

\subsection{EMC-direktiivi 89/336/ETY ja
2004/108/EY}

EMC-direktiivi 89/336/ETY määrittelee ainoastaan laitteistolta
vaadittavat ominaisuudet sähkömagneettisen yhteensopivuuden
takaamiseksi. Direktiivin tarkoitus on ohjeistaa valmistajia tekemään
elektromagneettisesti yhteensopivia laitteita. Direktiivi koskee kaikkia
sähkölaitteita ja -asennuksia, joita ei direktiivissä ole erikseen
rajattu sen ulkopuolelle {[}@89/336/ETY{]}. Direktiivi 2004/108/EY
kumosi vanhemman direktiivin 89/336/ETY 20.7.2004 alkaen. 2004/108/EY
mm. erotteli kiinteille asennuksille ja laitteille tehtävät asennukset,
sekä yksinkertaisti vaatimustenmukaisuuden arviointimenettelyä.
{[}@2004/108/EY{]}

\subsection{Radio- ja telepäätelaitedirektiivi
1999/5/EY}

Radio- ja telepäätelaitedirektiivi 1999/5/EY määrittää radio ja
telepäätelaitteiden yhteensopivuuden euroopan laajuisesti. Kaikkiin
direktiivin piiriin kuuluvien laitteiden tulee olla turvallisia
käyttäjälle ja muille henkilöille, sekä täyttää vaaditut
suojavaatimukset sähkömagneettisen yhteensopivuuden osalta. Lisäksi
direktiivi määrittää että laitteistojen tulee olla rakennettuja siten
että ne käyttävät tehokaasti radioviestintään varattua spektriä ja
resursseja. Tietyille laiteluokille on lisäksi määritelty vielä muita
vaadittuja lisäominaisuuksia, kuten yksityisyyden suojan takaamisen,
yhteensopivuuden muiden laitteistojen välillä, sisältävät petoksia
ehkäiseviä ominaisuuksia, tukevat hätäpalveluihin pääsyn takaavia
ominaisuuksia ja/tai sisältävät ominaisuuksia joilla laitteistojen
käyttö tehdään helpommaksi vammaisille {[}@1999/5/EY{]}. Direktiivi
1999/5/EY on kumottu 13.6.2016 alkaen direktiivillä 2014/53/EU
radiolaitteiden asettamista saataville markkinoilla koskevan
jäsenvaltioiden lainsäädännön yhdenmukaistamisesta. {[}@2014/53/EU{]}

\subsection{IP-suojaluokitus}

IP-suojaluokitus on standardissa IEC 60529 määritetty järjestelmä
sähkölaitteiden tiiveyden määrittämiseksi. IP-luokitus kertoo laitteiden
suojauksen pölyä ja vettä vastaan. (Maxim Integrated Products 2007)

\subsubsection{IP54}

IP54-suojaluokitetut tuotteet ovat pölysuojattuja (ei täydellistä
tiiveyttä, mutta ei pölykertymiä), sekä roiskesuojattuja.

\subsubsection{IP67/66}

IP67/66 -suojaluokitetut tuotteet ovat täysin pölytiiviitä ja kestävät
suurella paineella tulevan vesiruiskun. IP67/66-tuotteet kestävät
tärinää ja iskuja 5M3-vaatimusten mukaisesti. (DIN EN 60721-3-5, MIL-STD
810F.)


%----------------------------------------------------------------------------------------
%   BIBLIOGRAPHY 
%----------------------------------------------------------------------------------------

\IfLanguageName{finnish}{\bibliographystyle{vancouver_fi}}{\bibliographystyle{vancouver}}
%line space
%\singlespacing %removed otherwise the appendix are also single space
\begin{flushleft}
\begin{singlespacing}
\bibliography{motonkone_biblio}
\end{singlespacing}
\end{flushleft}

%for conting the pages
\label{LastPage}~


\end{document}
